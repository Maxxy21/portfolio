\documentclass{resume}
\usepackage[left=0.4 in,top=0.4in,right=0.4 in,bottom=0.4in]{geometry}
\newcommand{\tab}[1]{\hspace{.2667\textwidth}\rlap{#1}} 
\newcommand{\itab}[1]{\hspace{0em}\rlap{#1}}

\name{Maxwell Aboagye}
\address{Hamburg, Deutschland}
\address{
    \href{mailto:aboagyemaxwell@outlook.com}{aboagyemaxwell@outlook.com} \\
    \href{tel:+4917645744347}{+49 176 4574 4347}}

\begin{document}

\begin{rSection}{BERUFSERFAHRUNG}
\textbf{Business Intelligence Analyst} \hfill Oktober 2024 – Heute\\
Amazon \hfill \textit{Hamburg, Deutschland}
\begin{itemize}
    \item Entwicklung und Pflege von operativen Dashboards mit Power BI, Grafana und Amazon Quicksight
    \item Analyse von KPIs zur Leistungsmessung von Lieferpartnern (DSPs) im Bereich Sicherheit und Effizienz
    \item Automatisierung von ETL-Datenprozessen zur Unterstützung täglicher Entscheidungsfindung
    \item Zusammenarbeit mit Fachabteilungen zur Umsetzung von Reporting-Anforderungen
    \item Erstellung datenbasierter Management-Reports zur Prozessoptimierung
    \item Performance-Optimierung und Sicherstellung der Datenqualität in Reports
\end{itemize}

\textbf{Software Developer Praktikum} \hfill September 2022 – Juni 2023\\
ewo GmbH \hfill \textit{Bozen, Italien}
\begin{itemize}
    \item Entwicklung von Webanwendungen mit React.js und Spring Boot
    \item CI/CD-Implementation mit Jenkins und Docker
    \item Unterstützung bei Cloud-Deployments auf Azure
    \item Dokumentation von DevOps-Prozessen zur Effizienzsteigerung
\end{itemize}
\end{rSection}

\begin{rSection}{AUSBILDUNG}
{\bf Bachelor of Science in Informatik}, Freie Universität Bozen \hfill {2020 – 2025}\\
Schwerpunkte: Softwareentwicklung, Webtechnologien, Verteilte Systeme, Datenanalyse
\end{rSection}

\begin{rSection}{TECHNISCHE KENNTNISSE}
\begin{tabular}{ @{} >{\bfseries}l @{\hspace{6ex}} l }
BI \& Visualization & Tableau, Power BI, Amazon Quicksight, Excel \\
Programmiersprachen & SQL, Python, JavaScript, TypeScript, Java \\
Databases & PostgreSQL, MySQL, MS SQL Server \\
ETL \& Data Processing & ETL-Prozesse, Power Query, API-Anbindung, Data Warehousing \\
Frameworks & Spring Boot, React.js, Express.js, Next.js \\
Cloud \& DevOps & AWS (EC2, S3, Lambda), Azure, Docker, Jenkins, Git \\
\end{tabular}
\end{rSection}

\begin{rSection}{SCHLÜSSELPROJEKTE}
\textbf{DSP Management Extension (Firefox Add-on)}
\begin{itemize}
    \item Enterprise-Extension für Amazon DSP-Roster-Management
    \item Automatisierte Checks, Webhook-Integration, CI/CD-Pipeline mit GitHub Actions
    \item Vollständiger Software-Lebenszyklus: Build, Sign, Release, Deployment
\end{itemize}

\textbf{Pista - AI-Powered Startup Pitch Evaluation Platform}
\begin{itemize}
    \item Full-Stack-Webanwendung mit Multi-Provider-AI-Integration (GPT-4, Claude, Gemini)
    \item Next.js 15, PostgreSQL/Prisma, multimodale Eingabe mit Echtzeit-Transkription
    \item Interactive Q\&A-Engine und PDF-Export-Funktionalität für Investment-Reports
\end{itemize}
\end{rSection}

\begin{rSection}{KERNKOMPETENZEN}
\begin{itemize}
    \item \textbf{Business Intelligence:} Dashboard-Entwicklung, Datenvisualisierung, KPI-Analyse
    \item \textbf{Datenmodellierung:} ETL-Prozesse, Data Warehouse-Architekturen, SQL-Optimierung
    \item \textbf{Stakeholder-Management:} Anforderungserhebung, Reporting-Lösungen, Präsentationen
    \item \textbf{Agile Methoden:} Scrum, kontinuierliche Verbesserung, teamorientierte Entwicklung
    \item \textbf{Performance \& Qualität:} Usability-Optimierung, Datenqualitätssicherung, Best Practices
\end{itemize}
\end{rSection}

\begin{rSection}{SPRACHEN}
\begin{tabular}{ @{} l @{\hspace{8ex}} l }
Englisch (verhandlungssicher) & Deutsch (verhandlungssicher) \\
Italienisch (verhandlungssicher) & Asante Twi (Muttersprache) \\
\end{tabular}
\end{rSection}

\end{document}